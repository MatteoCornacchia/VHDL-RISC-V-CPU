\chapter{Conclusion}

In conclusion, the project has been successfully completed and the developed system operates as expected. Each main component of the project, namely the RISC-V CPU, the CORDIC module, and the Nexys 4 DDR board interface, functions correctly when considered individually. Moreover, the interaction between the CORDIC module and the Nexys 4 DDR platform has been validated through hardware tests, confirming the correct communication between computation logic and user interface.

From an educational perspective, the project can be considered a successful outcome, as it covers the design, verification, and hardware integration of multiple digital systems. Nevertheless, several limitations and possible future extensions can be identified:

\begin{itemize}
    \item The RISC-V CPU implements a basic single-cycle architecture without pipelining. Although the design is modular and can be extended, introducing a pipelined architecture would significantly improve performance.
    \item The CORDIC module currently operates as a standalone unit and is not integrated into the CPU datapath. A future extension could include the CORDIC as a dedicated functional unit or coprocessor within the CPU.
    \item The CORDIC algorithm exhibits limitations for specific edge cases, in particular for input angles equal to zero or exact multiples of $\pi/2$, where the sign of the computed sine and cosine may be incorrect.
    \item The interaction with the Nexys 4 DDR board is functional but not fully user-friendly, as the input angle must be provided in binary form using switches. Future improvements could focus on simplifying user input, for instance by introducing a more intuitive input method or a higher-level interface.
\end{itemize}

Overall, the project provides a solid foundation for further extensions and represents a meaningful learning experience in hardware design, digital architectures, and FPGA-based system integration.
